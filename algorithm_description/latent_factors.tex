\documentclass{llncs}
\usepackage{indentfirst}
\usepackage{url}

\begin{document}
\title{Prediction of Adverse Side-Effects using recommender-system}
\author{Diogo Pinto and Pedro Costa
\thanks{Special thanks to Prof. Vítor Santos Costa and Prof. Rui Camacho}}
\institute{INESC Tec.}

\maketitle


\begin{abstract}
Using one of the best algorithms in use for recommender systems, we find the most probable adverse side effects that a drug may have in addition to a set of ones previously detected. The efficiency and dependence on previously known adverse side effects of this algorithm may be further boosted by adding the most important chemical structures. 

This enables drugs industry to predict in side-effects even before the drug entering the market, saving millions of dollars in the process.
\end{abstract}

\section{Data}
ADReCS
\footnote{\url{http://bioinf.xmu.edu.cn/ADReCS}}
is a database, maintained by Xiamen University, of adverse drugs' reactions ontologies, that enables the standardization and hierarchization of ADR terms.

This database divides in two: one with the information on drugs and ADRs related, and another with ontologies about ADRs and drugs that may lead to a certain ADR. Both are originally kept in XML format, but were imported into an relational database for efficiency of querying.

\subsubsection{Drugs}

Each drug has:
\begin{itemize}
\item Id \textit{e.g.: BADD D00002}
\item Name \textit{e.g.: Abarelix}
\item Description
\item ATC (Anatomical Therapeutic Chemical classification system) \textit{e.g.: L02BX01}
\item Synonyms list \emph{e.g.: Plenaxis, Plenaxis depot, PPI-149, Abarelix-Depot-M.}
\item Indications
\item CAS (Chemical Abstracts Service registry number) \emph{e.g.: 183552-38-7}
\item ADR List. Each ADR has:
	\begin{itemize}
	\item Term \textit{e.g.: Hypersensitivity}
	\item Id \textit{e.g.: 10.01.03.003}
	\item Frequency \textit{e.g.: $>5\%$} (this information may be omitted)
	\end{itemize}
\end{itemize}

\subsubsection{ADRs}

Each ADR has:
\begin{itemize}
\item Id \emph{e.g.: 01.01.01.003}
\item Term \emph{e.g.: Factor I Deficiency}
\item Synonims List.
\item Description
\item Code Meddra \emph{e.g.: 10016075}
\item List of drugs that may lead to this ADR. Each drug entry has:
	\begin{itemize}
	\item Id \emph{e.g.: BADD D00937}
	\item Name \emph{e.g.: Oxytocin}
	\end{itemize}
\end{itemize}

\subsection{Data Structure}
The data structure used was derived from the ADReCS. From this database was built a matrix, with the \textbf{rows} being the drugs and \textbf{columns} being the ADRs.

For a given row (\emph{i.e.} drug), a value of 5 is given to a column if that drug has the corresponding ADR. 
The value 5 is used so that we are able to scale (\emph{i.e.} measure of its importance) the importance of the features that characterize the drugs. That enables that, in the future, if we add to the columns more features from a different nature (\emph{e.g.} main chemical substructures), it is possible to give a different scale, higher or lower than the first one.

\section{Singular Values Decomposition}
The singular value decomposition is a factorization of a real or complex matrix.

Formally, the singular value decomposition of an $m \times n$ real or complex matrix $M$ is a factorization of the form $M = U \Sigma V^\top$, where $U$ is an $m \times min(m, n)$ real or complex unitary matrix, $\Sigma$ is an $min(m, n) \times min(m, n)$ rectangular diagonal matrix with non-negative real numbers on the diagonal, and $V^\top$ (the conjugate transpose of $V$, or simply the transpose of $V$ if V is real) is an $min(m, n) \times n$ real or complex unitary matrix.
\footnote{\url{http://en.wikipedia.org/wiki/Singular_value_decomposition}}

\subsection{Dimensionality Reduction}
There is a technique which consists on removing the rows and columns from $U$, $\Sigma$ and $V$ that corresponds to the lower singular values (constant in the $\Sigma$ matrix), enabling to lower the dimension of the data while abstracting the features to a higher level. The criteria for stopping removing columns and rows is
$$\sum_{i=1}^{k} NewSigma[i,i]^2 \geq 0.9 \times \sum_{i=1}^{min(m, n)} OriginalSigma[i,i]^2$$

After this step, $M$ is a $m \times k$ matrix, $\Sigma$ is a $k \times k$ matrix, and $V$ is a $n \times k$, where $k < min(m, n)$.

It is possible to multiply the obtained matrices in the same way as before and identify that the obtained matrix is very similar to the original one.

\section{Gradient Descent}

After obtaining the SVD, we obtain $P = M \times \sqrt{\Sigma}$ and $Q = V \times \sqrt{\Sigma}$, so that we only have two matrices to work with, keeping all the information intact.

This two matrices are approximated to the original one using gradient descent, by minimizing the distance for each element of the $P \times Q^\top$ to its desired value.

It starts with a predefined learning rate, and if the distance from the obtained matrix from the original is greater than in the last iteration, it is lowered by a factor of 3.
\footnote{Although implemented, this step has only 2 iterations, as the previous step produces already a very optimized solution. By adding new features, it is possible that the relevance of this step is enhanced.}

\section{Resulting artifacts}
From the sequence of steps described above, we obtain two main matrices. For normalization, we have to perform
$$P = P \times \sqrt{\Sigma}^{-1}$$
and similarly for $Q$.

\section{Possible operations}
The following operations are possibly applied to the inverse cases.

\begin{itemize}
\item By computing the cosine distance between two rows (drugs), it is possible to determine the similarity between them;
\item Predict ADRs. By $Drug \times Q$, we obtain the latent factors for this drug, and, as such, $Drug \times Q \times Q^\top$ gives the prediction of ADRs for that drug.
\item By changing how many rows we remove in the dimensionality reduction step, it is possible to obtain clusters of ADRs (by selecting, for each latent factor of a given ADR, the one that presents the highest value), and next determine the most prominent ADR from the group (by selecting the ADR that has the highest latent value from them all)
\end{itemize}

\section{Testing}

\subsection{Real World Simulation}
For simulating what really happens, we provide a test set (independent from the training set), and remove randomly between $0.3$ and $0.7$ of its ADRs. By running this for a random row, for a given number of iterations...

\end{document}